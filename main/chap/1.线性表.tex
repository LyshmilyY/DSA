\chapterimage{chap39.png}
\chapter{线性表}

\section{线性表的定义和基本操作}
\begin{definition}[线性表]
    $$L=(a_{1},a_{2},a_{3},\dots,a_{i},a_{i+1},\dots,a_{n})$$
    \begin{enumerate}
        \item 有相同数据类型的 $n$ 个数据元素的有限序列
        \item 存在唯一的一个被称作“第一个”的数据元素
        \item 存在唯一的一个被称作“最后一个”的数据元素
        \item 除第一个外,每个数据元素有且仅有一个直接前驱
        \item 除最后一个外,每个数据元素有且仅有一个直接后继
    \end{enumerate}
\end{definition}

\begin{definition}[线性表基本操作]
    \begin{enumerate}
        \item $\mathbf{InitList(\& L)}$  初始化线性表,构造一个空的线性表 $L$
        \item $\mathbf{DestroyList(\& L)}$  销毁线性表,销毁线性表 $L$
        \item $\mathbf{ListInsert(\&L,i,e)}$ 插入操作,在线性表 $L$ 的第 $i$ 个位置插入元素 $e$
        \item $\mathbf{ListDelete(\&L,i,\&e)}$ 删除操作,删除线性表 $L$ 的第 $i$ 个位置的元素,并用 $e$ 返回其值
        \item $\mathbf{GetElem(L,i,\&e)}$ 取值操作,返回线性表 $L$ 的第 $i$ 个位置的元素
        \item $\mathbf{LocateElem(L,e)}$ 定位操作,在线性表 $L$ 中查找与给定值 $e$ 相等的元素
        \item $\mathbf{Length(L)}$ 求表长,返回线性表 $L$ 的元素个数
        \item $\mathbf{Empty(L)}$ 判空操作,若 $L$ 为空表,则返回 $\text{TRUE}$,否则返回 $\text{FALSE}$
        \item $\mathbf{PrintList(L)}$ 输出操作,按顺序输出线性表 $L$ 的所有元素 
    \end{enumerate}
\end{definition}

\section{线性表的顺序表示}
\subsection{顺序表}
\begin{definition}[顺序表]
    \begin{enumerate}
        \item 用顺序存储的方式实现线性表
        \item 顺序存储: 把逻辑上相邻的元素存储在物理上相邻的存储单元中,元素之间的关系由存储单元之间的邻接关系来体现
    \end{enumerate}
\end{definition}

\begin{macbox}{common.h}
	\begin{minted}{c}
#include <assert.h>
#include <stdbool.h>
#include <stdio.h>
#include <stdlib.h>
#include <string.h>
#include <time.h>
#include <math.h>
#define OK 1
#define ERROR 0
#define OVERFLOW -2

typedef int Status;

// 定义数据域
typedef struct elem
{
    char name[20];
    int Math;
    int English;
    int Politics;
    int Computer; 
}elem;
    \end{minted}
\end{macbox}

\begin{macbox}{SqList.h}
	\begin{minted}{c}
#define INITSIZE 10
// 定义顺序表
typedef struct SqList
{
    int length;
    int MaxSize;
    elem* data;
}SqList;

// 函数声明 

Status InitList(SqList * L); // 初始化
Status IncreaseSize(SqList * L, int len); // 增加表长
Status ListInsert(SqList * L, elem e, int i); // 插入
Status ListDelete(SqList * L, int i, elem * e); // 删除
int LocateElem(SqList L, elem e); // 按值查找
Status GetElem(SqList L, int i, elem * e); // 按位查找 
int Length(SqList L); // 表长
Status Empty(SqList L);// 判空
void PrintList(SqList L); // 打印
void Printelem(elem e); // 打印单个数据元素
    \end{minted}
\end{macbox}


\subsubsection{初始化顺序表}

\begin{macbox}{InitList}
	\begin{minted}{c}
// 初始化
Status InitList(SqList * L)
{
    L->data = malloc(INITSIZE * sizeof(elem));
    if (!L->data) exit(OVERFLOW);
    L->length = 0;
    return OK;
}
    \end{minted}
\end{macbox}


\subsubsection{增加顺序表}

\begin{macbox}{IncreaseSize}
    \begin{minted}{c}
// 增加表长
Status IncreaseSize(SqList * L, int len)
{
    elem *p = L->data;
    L->data = (elem *)malloc((L->MaxSize + len) * sizeof(elem));
    if (!L->data) exit(OVERFLOW);
    for (int i = 0; i < L->length; i++)
    {
        L->data[i] = p[i];
    }
    L->MaxSize += len;
    free(p);
    return OK;
}
    \end{minted}
\end{macbox}


\subsubsection{插入顺序表}

\begin{macbox}{ListInsert}
    \begin{minted}{c}
// 插入
Status ListInsert(SqList * L, int i, elem e)
{
    if (i < 1 || i > L->length + 1) return ERROR;
    if (L->length >= L->MaxSize) IncreaseSize(L, INITSIZE);
    for (int j = L->length - 1; j >= i-1; j--)
    {
        L->data[j + 1] = L->data[j];
    }
    L->data[i - 1] = e;
    L->length++;
    return OK;
}
    \end{minted}
\end{macbox}

\subsubsection{删除顺序表}

\begin{macbox}{ListDelete}
    \begin{minted}{c}
// 删除
Status ListDelete(SqList * L, int i, elem * e)
{
    if (i < 1 || i > L->length) return ERROR;
    *e = L->data[i - 1];
    for (int j = i; j < L->length; j++)
    {
        L->data[j - 1] = L->data[j];
    }
    L->length--;
    return OK;
}
    \end{minted}
\end{macbox}

\subsubsection{顺序表查找}

\begin{macbox}{LocateElem \& GetElem}
    \begin{minted}{c}
// 按值查找
int LocateElem(SqList L, elem e)
{
    for (int i = 0; i < L.length; i++)
    {
        if (L.data[i].name == e.name) 
            return i + 1;
    }
    return ERROR;
}

// 按位查找
Status GetElem(SqList L, int i, elem * e)
{
    if (i < 1 || i > L.length) return ERROR;
    *e = L.data[i - 1];
    return OK;
}
    \end{minted}
\end{macbox}

\subsubsection{顺序表辅助函数}

\begin{macbox}{Length \& Empty \& PrintList}
    \begin{minted}{c}
// 表长
int Length(SqList L)
{
    return L.length;
}
// 判空
Status Empty(SqList L)
{
    if (L.length == 0) return OK;
    else return ERROR;
}
// 打印
void PrintList(SqList L)
{
    printf("%5s %15s %15s %15s %20s\n",
    "Name", "Score:Math", "Score:English", 
    "Score:Politics", "Score:Computer");
    for (int i = 0; i < L.length; i++)
    {
        printf("%5s %10d %15d %15d %20d\n",
        L.data[i].name,L.data[i].Math,L.data[i].English,
        L.data[i].Politics,L.data[i].Computer);
    }
    printf("\n");
} 

void Printelem(elem e)
{
    printf("%5s %15s %15s %15s %20s\n",
    "Name", "Score:Math", "Score:English", 
    "Score:Politics", "Score:Computer");
    printf("%5s %10d %15d %15d %20d\n",
    e.name,e.Math,e.English,e.Politics,e.Computer);
    printf("\n");
}
    \end{minted}
\end{macbox}

\subsubsection{顺序表实例化}
\begin{macbox}{EgSqList.c}
    \begin{minted}{c}
int main()
{
    SqList L;
    elem student;
    InitList(&L);
    for (int i = 0; i < 5; i++)
    {
        scanf("%s %d %d %d %d",student.name, &(student.Math), 
        &(student.English),&(student.Politics),&(student.Computer));
        ListInsert(&L,i+1,student);
    }
    printf("初始顺序表:\n");
    PrintList(L);
    strcpy(student.name,"LY");
    student.Math = 150;
    student.English = 85;
    student.Politics = 75;
    student.Computer = 145;
    ListInsert(&L,2,student);
    printf("在第二个位置插入新元素:\n");
    PrintList(L);
    ListDelete(&L,3,&student);
    printf("删除第三个元素:\n");
    Printelem(student);
    printf("删除后的顺序表:\n");
    PrintList(L);
    return 0;
}
    \end{minted}
\end{macbox}

\begin{macbox}{data.txt}
    \begin{minted}{shell}
Amy 123 86 74 143
Bob 135 75 81 108
Dav 118 74 80 134
Fab 98  56 67 120
Joy 108 64 70 118
    \end{minted}
\end{macbox}

\begin{macbox}{output}
    \begin{minted}{shell}
./EgSqList < data.txt
初始顺序表:
Name      Score:Math   Score:English  Score:Politics       Score:Computer
Amy        123              86              74                  143
Bob        135              75              81                  108
Dav        118              74              80                  134
Fab         98              56              67                  120
Joy        108              64              70                  118

在第二个位置插入新元素:
Name      Score:Math   Score:English  Score:Politics       Score:Computer
Amy        123              86              74                  143
LY         150              85              75                  145
Bob        135              75              81                  108
Dav        118              74              80                  134
Fab         98              56              67                  120
Joy        108              64              70                  118
    \end{minted}
\end{macbox}
\begin{macbox}{output}
    \begin{minted}{shell}
删除第三个元素:
Name      Score:Math   Score:English  Score:Politics       Score:Computer
Bob        135              75              81                  108

删除后的顺序表:
Name      Score:Math   Score:English  Score:Politics       Score:Computer
Amy        123              86              74                  143
LY         150              85              75                  145
Dav        118              74              80                  134
Fab         98              56              67                  120
Joy        108              64              70                  118
    \end{minted}
\end{macbox}
\section{线性表的链式表示}
\subsection{单链表}
\begin{definition}[单链表]
    \begin{enumerate}
        \item 单链表第一个元素在删除和增加需要特殊处理
        \item 头插法和尾插法的区别, 头插法实现反置
        \item 在增删、初始化操作传入引用, 在判空、打印以及求表长传入值
    \end{enumerate}
\end{definition}
\begin{macbox}{LNode}
	\begin{minted}{c}
typedef struct LNode
{
    elem data;
    struct LNode *next;
}LNode,*LinkList;
    \end{minted}
\end{macbox}

\begin{macbox}{SlList.h}
    \begin{minted}{c}
// 函数声明 
Status InitList(LinkList *L); // 初始化
Status Empty(LinkList L);// 判空
LNode * LocateElem(LinkList L, elem e); // 按值查找
LNode * GetElem(LinkList *L, int i); // 按位查找
Status InsertNextNode(LNode *p,elem e); //指定元素后插入
Status InsertPreNode(LNode *p,elem e); //指定元素前插入
Status ListInsert(LinkList *L, int i, elem e); // 按位置插入
Status ListDelete(LinkList *L, int i, elem * e); // 按位置删除
Status DeleteNode(LNode *p); // 删除指定节点
int Length(LinkList L); // 表长
void PrintList(LinkList L); // 打印链表
void PrintNode(LNode p); // 打印节点
LinkList TailInsert(LinkList *L,int n); // 尾插法
LinkList HeadInsert(LinkList *L,int n); // 头插法
void ListReverse(LinkList *L); //翻转链表
    \end{minted}
\end{macbox}

\subsubsection{初始化单链表}
\begin{macbox}{InitList}
	\begin{minted}{c}
// 初始化
Status InitList(LinkList *L)
{
    *L = (LNode *) malloc (sizeof(LNode));
    if (*L == NULL)
        return ERROR;
    (*L) ->next = NULL;
    return OK;
}
// 判空
Status Empty(LinkList L)
{
    if (L->next == NULL)
        return OK;
    return ERROR;
}
    \end{minted}
\end{macbox}

\subsubsection{单链表查找}
\begin{macbox}{Find}
	\begin{minted}{c}
// 按值查找
LNode * LocateElem(LinkList L, elem e)
{
    LNode *p = L->next;
    while(p != NULL && p->data.name != e.name)
        p = p->next;
    return p;
}
// 按位查找 
LNode * GetElem(LinkList *L, int i)
{
    if (i < 1)
        return NULL;
    LNode *p = *L;
    // 第 j 个节点 p
    int j = 0;
    // 找到第 i 个元素的位置
    while (p !=NULL && j < i)
    {
        p = p->next;
        j++;
    }
    return p;
}
    \end{minted}
\end{macbox}
\subsubsection{单链表插值}
\begin{macbox}{Insert 辅助函数}
	\begin{minted}{c}
//指定元素后插入
Status InsertNextNode(LNode *p,elem e)
{
    LNode *s = (LNode *)malloc(sizeof(LNode));
    if (s == NULL || p == NULL)
        return ERROR;
    s->data = e;
    s->next = p->next;
    p->next = s;
    return OK;
}
//指定元素前插入
Status InsertPreNode(LNode *p,elem e)
{
    LNode *s = (LNode *)malloc(sizeof(LNode));
    if (s == NULL || p == NULL)
        return ERROR;
    s->next = p->next;
    p->next = s;
    s->data = p->data;
    p->data = e;
    return OK;
}
    \end{minted}
\end{macbox}

\begin{macbox}{ListInsert}
	\begin{minted}{c}
// 按位置插入
Status ListInsert(LinkList *L, int i, elem e)
{
    if (i < 1)
        return ERROR;
    LNode *p = (LNode *)malloc(sizeof(LNode));
    if (p == NULL)
        return ERROR;
    if (i ==1)
        {
            p->data = e;
            p->next = (*L)->next;
            (*L)->next = p;
            return OK;
        }
    p = GetElem(L,i-1);
    if (p == NULL)
        return ERROR;
    return InsertNextNode(p,e);
}
    \end{minted}
\end{macbox}
\subsubsection{单链表删除}
\begin{macbox}{ListDelete}
	\begin{minted}{c}
// 删除
Status ListDelete(LinkList *L, int i, elem * e)
{
    if (i < 1)
        return ERROR;
    if (i == 1)
    {
        if ((*L)->next == NULL)
            return ERROR;
        *e = (*L)->next->data;
        (*L)->next = (*L)->next->next;
        return OK;
    }
    LNode *p = GetElem(L,i-1);
    if (p == NULL || p->next == NULL)
        return ERROR;
    LNode * q = p->next;
    *e = q->data;
    p->next =q->next;
    //释放删除节点空间
    free(q);
    return OK;
}
    \end{minted}
\end{macbox}
\subsubsection{单链表辅助函数}
\begin{macbox}{Length \& PrintList}
	\begin{minted}{c}
// 表长
int Length(LinkList L)
{
    int len =0;
    LNode *p = L;
    while(p->next != NULL)
    {   p = p->next;
        len++;}
    return len;
}
// 打印
void PrintList(LinkList L)
{
    printf("%5s %15s %15s %15s %20s\n",
    "Name", "Score:Math", "Score:English", 
    "Score:Politics", "Score:Computer");
    LNode *p = L->next;
    while(p != NULL)
    {   printf("%5s %10d %15d %15d %20d\n",
        p->data.name,p->data.Math,p->data.English,
        p->data.Politics,p->data.Computer);
        p = p->next;}
    printf("\n");
}
// 打印单个数据元素
void PrintNode(LNode p)
{
    printf("%5s %15s %15s %15s %20s\n","Name", 
    "Score:Math", "Score:English",
     "Score:Politics", "Score:Computer");
    printf("%5s %10d %15d %15d %20d\n",
    p.data.name,p.data.Math,p.data.English,
    p.data.Politics,p.data.Computer);
    printf("\n");
}
    \end{minted}
\end{macbox}
\subsubsection{头插法 \& 尾插法}
\begin{macbox}{TailInsert \& HeadInsert}
	\begin{minted}{c}
// 尾插法实现单链表
LinkList TailInsert(LinkList *L,int n)
{
    LNode *r = *L;
    for (int i = 0; i < n; i++)
    {
        LNode *p = (LNode*)malloc(sizeof(LNode));
        scanf("%s %d %d %d %d\n",
        p->data.name,&(p->data.Math),&(p->data.English),
        &(p->data.Politics),&(p->data.Computer));
        p->next = r->next;
        r->next = p;
        r = p;
    }
    return *L;
}
// 头插法实现单链表
LinkList HeadInsert(LinkList *L,int n)
{
    for (int i = 0; i < n; i++)
    {
        LNode *p = (LNode*)malloc(sizeof(LNode));
        scanf("%s %d %d %d %d\n",
        p->data.name,&(p->data.Math),&(p->data.English),
        &(p->data.Politics),&(p->data.Computer));
        p->next = (*L)->next;
        (*L)->next = p;
    }
    return *L;
}
    \end{minted}
\end{macbox}

\begin{macbox}{ListReverse}
	\begin{minted}{c}
// 反转单链表
void ListReverse(LinkList *L)
{
    LNode *p,*q;
    p = (*L)->next;
    (*L)->next = NULL;
    while(p != NULL)
    {
        q = p->next;
        p->next = (*L)->next;
        (*L)->next = p;
        p = q;
    }
}
    \end{minted}
\end{macbox}
\subsubsection{单链表实例化}
\begin{macbox}{EgSlList}
	\begin{minted}{c}
int main()
{
    LinkList L;
    LNode p;
    elem student;
    InitList(&L);
    printf("尾插法建立单链表:\n");
    TailInsert(&L,5);
    PrintList(L);
    printf("反转单链表:\n");
    ListReverse(&L);
    PrintList(L);
    printf("删除第三个元素后的单链表:\n");
    ListDelete(&L,3,&student);
    PrintList(L);
    printf("被删除的元素:\n");
    p.data = student;
    PrintNode(p);
    printf("单链表第一个位置插入:\n");
    ListInsert(&L,1,student);
    PrintList(L);
    return 0;
}
    \end{minted}
\end{macbox}

\begin{macbox}{output}
    \begin{minted}{shell}
./SlList < data.txt
尾插法建立单链表:
Name    Score:Math   Score:English  Score:Politics    Score:Computer
Amy        123            86              74               143
Bob        135            75              81               108
Dav        118            74              80               134
Fab         98            56              67               120
Joy        108            64              70               118

反转单链表
Name    Score:Math   Score:English  Score:Politics    Score:Computer
Joy        108            64              70               118
Fab         98            56              67               120
Dav        118            74              80               134
Bob        135            75              81               108
Amy        123            86              74               143
    \end{minted}
\end{macbox}
\begin{macbox}{output}
    \begin{minted}{shell}
删除第三个元素后的单链表:
Name    Score:Math   Score:English  Score:Politics    Score:Computer
Joy        108            64              70               118
Fab         98            56              67               120
Bob        135            75              81               108
Amy        123            86              74               143

被删除的元素:
Name    Score:Math   Score:English  Score:Politics    Score:Computer
Dav        118            74              80               134

单链表第一个位置插入:
Name    Score:Math   Score:English  Score:Politics    Score:Computer
Dav        118            74              80               134
Joy        108            64              70               118
Fab         98            56              67               120
Bob        135            75              81               108
Amy        123            86              74               143
    \end{minted}
\end{macbox}
\subsection{双链表}
\begin{definition}[双链表]
    \begin{enumerate}
        \item 双链表有两个指针域,分别指向直接前驱和直接后继
        \item 头插法和尾插法的区别, 头插法实现反置
        \item 在插入和删除时,第一个和最后一个节点有特殊情况
    \end{enumerate}
\end{definition}
\begin{macbox}{DNode}
	\begin{minted}{c}
typedef struct DNode
{
    elem data;
    struct DNode *prior;
    struct DNode *next;
}DNode,*DLinkList;
    \end{minted}
\end{macbox}

\begin{macbox}{DlList.h}
    \begin{minted}{c}
// 函数声明 
Status InitList(DLinkList *L); // 初始化
Status Empty(DLinkList L);// 判空
DNode * LocateElem(DLinkList L, elem e); // 按值查找
DNode * GetElem(DLinkList *L, int i); // 按位查找
Status InsertNextNode(DNode *p,elem e); //指定元素后插入
Status InsertPreNode(DNode *p,elem e); //指定元素前插入
Status ListInsert(DLinkList *L, int i, elem e); // 按位置插入
Status ListDelete(DLinkList *L, int i, elem * e); // 删除
Status DeleteNode(DNode *p); // 删除指定节点
int Length(DLinkList L); // 表长
void PrintList(DLinkList L); // 打印链表
void PrintNode(DNode p); // 打印节点
DLinkList TailInsert(DLinkList *L,int n); // 尾插法
DLinkList HeadInsert(DLinkList *L,int n); // 头插法
void ListReverse(DLinkList *L); //翻转链表
    \end{minted}
\end{macbox}

\subsubsection{初始化双链表}
\begin{macbox}{InitList}
	\begin{minted}{c}
// 初始化
Status InitList(DLinkList *L)
{
    (*L) = (DNode *)malloc(sizeof(DNode));
    if ((*L) == NULL)
        return ERROR;
    (*L)->prior = NULL;
    (*L)->next = NULL;
    return OK;
}
// 判空
Status Empty(DLinkList L)
{
    if (L->next == NULL)
        return OK;
    return ERROR;
}
    \end{minted}
\end{macbox}

\subsubsection{双链表查找}
\begin{macbox}{Find}
	\begin{minted}{c}

// 按值查找
DNode * LocateElem(DLinkList L, elem e)
{
    DNode *p = L->next;
    while (p != NULL)
    {
        if (p->data.name == e.name)
            return p;
        p = p->next;
    }
    return p;

}
// 按位查找
DNode * GetElem(DLinkList *L, int i)
{
    DNode * p = (*L);
    int j = 0;
    while (p->next != NULL && j<i)
    {
        p = p->next;
        j++;
    }
    return p;
}  
    \end{minted}
\end{macbox}

\subsubsection{双链表插值}
\begin{macbox}{Insert 辅助函数}
	\begin{minted}{c}
// 指定元素后插入
Status InsertNextNode(DNode *p,elem e)
{
    DNode *q = (DNode*)malloc(sizeof(DNode));
    if (q == NULL || p == NULL)
        return ERROR;
    q->data = e;
    p->next->prior = q;
    q->next = p->next;
    q->prior = p;
    p->next = q;
    return OK;
}
//指定元素前插入
Status InsertPreNode(DNode *p,elem e)
{
    DNode *q = (DNode*)malloc(sizeof(DNode));
    if (q == NULL || p ==NULL)
        return ERROR;
    q->data = e;
    p->prior->next = q;
    q->prior = p->prior;
    q->next = p;
    p->prior = q;
    return OK;
}
    \end{minted}
\end{macbox}

\begin{macbox}{ListInsert}
	\begin{minted}{c}
// 按位置插入
Status ListInsert(DLinkList *L, int i, elem e)
{
    if (i < 1)
        return ERROR;
    DNode *p = (DNode*)malloc(sizeof(DNode));
    if (p == NULL)
        return ERROR;
    if (i == 1)
    {
        p->data = e;
        p->next = (*L)->next;
        (*L)->next->prior = p;
        (*L)->next = p;
        p->prior = (*L);
        return OK;
    }
    p = GetElem(L,i-1);
    return InsertNextNode(p,e);

} 
    \end{minted}
\end{macbox}
\subsubsection{双链表删除}
\begin{macbox}{ListDelete}
	\begin{minted}{c}
// 删除
Status ListDelete(DLinkList *L, int i, elem * e)
{
    if (i < 1)
        return ERROR;
    DNode *p = (*L)->next;
    if (i == 1)
    {
        if (p == NULL)
            return ERROR;
        (*L)->next = p->next;
        *e = p->data;
        if (p->next != NULL)
            p->next->prior = (*L);
        free(p);
        return OK;
    }
    p = GetElem(L,i);
    if (p == NULL)
        return ERROR;
    *e = p->data;
    return DeleteNode(p);
}
    \end{minted}
\end{macbox}
\subsubsection{双链表辅助函数}
\begin{macbox}{Length}
	\begin{minted}{c}
// 表长
int Length(DLinkList L)
{
    int len = 0;
    while (L->next != NULL)
    {
        len++;
        L = L->next;
    }
    return len;
}
    \end{minted}
\end{macbox}

\begin{macbox}{PrintList}
	\begin{minted}{c}
// 打印链表
void PrintList(DLinkList L)
{
    printf("%5s %15s %15s %15s %20s\n",
    "Name", "Score:Math", "Score:English", 
    "Score:Politics", "Score:Computer");
    DNode *p = L->next;
    while(p != NULL)
    {
        printf("%5s %10d %15d %15d %20d\n",
        p->data.name,p->data.Math,p->data.English,
        p->data.Politics,p->data.Computer);
        p = p->next;
    }
    printf("\n");
}
// 打印节点
void PrintNode(DNode p)
{
    printf("%5s %15s %15s %15s %20s\n",
    "Name", "Score:Math", "Score:English", 
    "Score:Politics", "Score:Computer");
    printf("%5s %10d %15d %15d %20d\n",
    p.data.name,p.data.Math,p.data.English,
    p.data.Politics,p.data.Computer);
    printf("\n");
}
    \end{minted}
\end{macbox}
\subsubsection{头插法 \& 尾插法}
\begin{macbox}{TailInsert}
	\begin{minted}{c}
// 尾插法
DLinkList TailInsert(DLinkList *L,int n)
{
    DNode *r = *L;
    for (int i = 0; i < n; i++)
    {
        elem e;
        scanf("%s %d %d %d %d\n",
        e.name,&(e.Math),&(e.English),&(e.Politics),&(e.Computer));
        DNode *p = (DNode*)malloc(sizeof(DNode));
        if (p == NULL)
            return NULL;
        p->data = e;
        p->next = r->next;
        r->next = p;
        p->prior = r;
        r = p; 
    }
    return *L;
}
    \end{minted}
\end{macbox}

\begin{macbox}{HeadInsert}
	\begin{minted}{c}
// 头插法
DLinkList HeadInsert(DLinkList *L,int n)
{
    for (int i = 0; i < n; i++)
    {
        elem e;
        scanf("%s %d %d %d %d\n",
        e.name,&(e.Math),&(e.English),&(e.Politics),&(e.Computer));
        DNode *p = (DNode*)malloc(sizeof(DNode));
        if (p == NULL)
            return NULL;
        p->data = e;
        p->next = (*L)->next;
        p->prior = (*L);
        if ((*L)->next != NULL)
            (*L)->next->prior = p;
    }
    return *L;
}
    \end{minted}
\end{macbox}
\begin{macbox}{ListReverse}
	\begin{minted}{c}
//翻转链表
void ListReverse(DLinkList *L)
{
    DNode *p,*q;
    p = (*L)->next;
    (*L)->next = NULL;
    while(p != NULL)
    {
        q = p->next;
        p->next = (*L)->next;
        p->prior = (*L);
        if ((*L)->next != NULL)
            (*L)->next->prior = p;
        (*L)->next = p;
        p = q;
    }
}   
    \end{minted}
\end{macbox}
\subsubsection{双链表实例化}
\begin{macbox}{EgDlList}
	\begin{minted}{c}
int main()
{
    DLinkList L;
    DNode p;
    elem student;
    InitList(&L);
    printf("尾插法建立双链表:\n");
    TailInsert(&L,5);
    PrintList(L);
    printf("反转双链表:\n");
    ListReverse(&L);
    PrintList(L);
    printf("删除第三个元素后的双链表:\n");
    ListDelete(&L,3,&student);
    PrintList(L);
    printf("被删除的元素:\n");
    p.data = student;
    PrintNode(p);
    printf("双链表第一个位置插入:\n");
    ListInsert(&L,1,student);
    PrintList(L);
    return 0;
}
    \end{minted}
\end{macbox}

\begin{macbox}{output}
    \begin{minted}{shell}
./DlList < data.txt
尾插法建立双链表:
Name    Score:Math   Score:English  Score:Politics    Score:Computer
Amy        123            86              74               143
Bob        135            75              81               108
Dav        118            74              80               134
Fab         98            56              67               120
Joy        108            64              70               118

反转单链表
Name    Score:Math   Score:English  Score:Politics    Score:Computer
Joy        108            64              70               118
Fab         98            56              67               120
Dav        118            74              80               134
Bob        135            75              81               108
Amy        123            86              74               143
    \end{minted}
\end{macbox}
\begin{macbox}{output}
    \begin{minted}{shell}
删除第三个元素后的单链表:
Name    Score:Math   Score:English  Score:Politics    Score:Computer
Joy        108            64              70               118
Fab         98            56              67               120
Bob        135            75              81               108
Amy        123            86              74               143

被删除的元素:
Name    Score:Math   Score:English  Score:Politics    Score:Computer
Dav        118            74              80               134

单链表第一个位置插入:
Name    Score:Math   Score:English  Score:Politics    Score:Computer
Dav        118            74              80               134
Joy        108            64              70               118
Fab         98            56              67               120
Bob        135            75              81               108
Amy        123            86              74               143
    \end{minted}
\end{macbox}
\subsection{循环链表}
\begin{definition}[循环链表]
    \begin{enumerate}
        \item 分为循环单链表和循环双链表, 与单链表和双链表的区别在于尾节点指向头节点
        \item 判空操作与单链表、双链表存在差异
        \item 插入、删除、以及判断表头、表尾节点有一定变化
    \end{enumerate}
\end{definition}

\subsection{静态链表}
\begin{definition}[静态链表]
    \begin{enumerate}
        \item 使用数组的链表, 数组第一个元素当做头节点, 每个位置除了保存数据元素外, 还保存下一个元素的位置
        \item 初始化时将所有位置的下一个元素位置清空
        \item 插入、删除操作需要找到上一个节点的位置, 然后将下一个元素位置清空
    \end{enumerate}
\end{definition}
\section{Summary}
\subsection{顺序表与链表比较}
\begin{table}[ht]
    \centering
    \caption{顺序表与链表比较}
    \label{table: 顺序表与链表比较}
    \begin{tblr}{
        width = 0.8\textwidth,
        hlines = {1pt},
        hline{1,Z} = {2pt},
        vline{2-Y} = {1pt},
        cell{1}{1} = {r=1,c=2}{c},
        cell{1}{3} = {r=1,c=2}{c,teal7},
        cell{2}{1} = {r=2,c=1}{c},
        cell{2}{2} = {c},
        cell{2}{3} = {r=1,c=2}{l},
        cell{3}{2} = {c},
        cell{3}{3} = {r=1,c=2}{l},
        cell{4}{1} = {r=2,c=1}{c},
        cell{4}{2} = {c},
        cell{4}{3} = {r=1,c=2}{l},
        cell{5}{2} = {c},
        cell{5}{3} = {r=1,c=2}{l},
        cell{6}{1} = {r=2,c=2}{c},
        cell{6}{3} = {r=2,c=2}{l},
    }
    \diagbox{比较项目}{存储结构} &           & 顺序表                                                                  & \\
    空间                       & 存储空间   & {① 预先分配\\  ② 会导致空间闲置或溢出}                                      & \\
                              & 存储密度   & {① $\text{存储密度}=1$\\  ② 不需要额外空间来表示节点的逻辑关系}                & \\
    时间                       & 存取元素   &  {① 随机存取\\ ② 按位置访问元素时间复杂度 $O(1)$}                            & \\
                              & 插入、删除  &  平均移动一半的元素, 时间复杂度 $O(n)$                                      & \\
    适用情况                   &            & {① 表长变化不大,确定变化范围\\ ② 很少进行插入或删除, 经常按照元素位置序号访问数据} & \\
                              &           &                                                                         &\\
    \end{tblr}
    \myspace{1}
    \begin{tblr}{
        width = 0.8\textwidth,
        hlines = {1pt},
        hline{1,Z} = {2pt},
        vline{2-Y} = {1pt},
        cell{1}{1} = {r=1,c=2}{c},
        cell{1}{3} = {r=1,c=2}{c,blue7},
        cell{2}{1} = {r=2,c=1}{c},
        cell{2}{2} = {c},
        cell{2}{3} = {r=1,c=2}{l},
        cell{3}{2} = {c},
        cell{3}{3} = {r=1,c=2}{l},
        cell{4}{1} = {r=2,c=1}{c},
        cell{4}{2} = {c},
        cell{4}{3} = {r=1,c=2}{l},
        cell{5}{2} = {c},
        cell{5}{3} = {r=1,c=2}{l},
        cell{6}{1} = {r=2,c=2}{c},
        cell{6}{3} = {r=2,c=2}{l},
    }
    \diagbox{比较项目}{存储结构} &           & 链表                                                            & \\
    空间                       & 存储空间   & {① 动态分配 \\  ② 不会出现空间闲置或溢出}                            & \\
                              & 存储密度   & {① $\text{存储密度}<1$\\ ② 需要借助指针来表示节点的逻辑关系}          &\\
    时间                       & 存取元素   & {① 顺序存取\\ ② 按位置访问元素时间复杂度 $O(n)$}                     &\\
                              & 插入、删除  & 不需要移动元素, 确定插入或删除位置后,时间复杂度 $O(n)$                 &\\
    适用情况                   &            &{① 表长变化很大\\ ② 频繁进行插入或删除操作}                            &\\
                              &           &                                                                 &\\
    \end{tblr}
\end{table}

\subsection{单链表、双链表、循环链表比较}
\begin{table}[ht]
    \centering
    \caption{单链表、双链表、循环链表比较}
    \label{table: 单链表、双链表、循环链表比较}
    \small
    \begin{tblr}[m]{
        width = \textwidth,
        colsep = {0.5pt},
        hlines = {1pt},
        hline{1,Z} = {2pt},
        vline{2-Y} = {1pt},
        cells = {c},
    }
    \diagbox{链表名称}{操作名称}       & {查找\\表头节点}                                & {查找\\表尾节点}                                           & {查找节点 $\mathbf{*p}$\\ 前驱节点} \\
    {带头节点\\单链表 $L$}             & {$\mathbf{L->next}$\\ 时间复杂度 $O(1)$}   & {从 $\mathbf{L->next}$ 依次向后遍历\\ 时间复杂度 $O(n)$} & 无法找到前驱节点\\
    {带头节点头指针 $L$\\ 循环单链表}   &  {$\mathbf{L->next}$\\ 时间复杂度 $O(1)$}   & {从 $\mathbf{L->next}$ 依次向后遍历\\ 时间复杂度 $O(n)$} & {$\mathbf{p->next}$可以找到前驱节点\\ 时间复杂度 $O(n)$}\\
    {带头节点尾指针 $R$\\ 循环单链表}   & {$\mathbf{R->next}$\\ 时间复杂度 $O(1)$}    &  {$\mathbf{R}$\\ 时间复杂度 $O(1)$}                    & {$\mathbf{p->next}$可以找到前驱节点\\ 时间复杂度 $O(n)$}\\
    {带头节点\\双向循环链表 $L$}       & {$\mathbf{L->next}$\\ 时间复杂度 $O(1)$}    &  {$\mathbf{L->prior}$\\ 时间复杂度 $O(1)$}             & {$\mathbf{p->prior}$可以找到前驱节点\\ 时间复杂度 $O(1)$} \\
    \end{tblr}
\end{table}
