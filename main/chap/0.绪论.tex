\chapterimage{chapter 01.png}
\chapter{绪论}
\section{基本概念}
\begin{definition}[数据结构]
    \begin{enumerate}
        \item 数据结构是相互之间存在一种或多种特定关系的数据元素的集合
        \item 数据元素之间的关系可以是线性的、非线性的
        \item 数据结构三要素: 逻辑结构、存储结构(实现)、数据运算
        \item 算法: 有穷性、确定性、可行性、健壮性
    \end{enumerate}
\end{definition}
\begin{definition}[线性表]
    $$L=(a_{1},a_{2},a_{3},\dots,a_{i},a_{i+1},\dots,a_{n})$$
    \begin{enumerate}
        \item 有相同数据类型的 $n$ 个数据元素的有限序列
        \item 需要在原数据上进行修改
        \item 初始化、插入、删除(传地址); 长度、判空、打印(传值)
    \end{enumerate}
\end{definition}
线性表基本操作
\begin{macbox}{Lyshmily.Y}
    \begin{minted}{c}
InitList(&L) //初始化表,分配内存空间
DestroyList(&L) //销毁线性表,释放内存空间
ListInsert(&L,i,e) //在表L中第i个位置插入元素e
ListDelete(&L,i,&e) //删除表L中第i个位置的元素,并用e返回删除元素的值
LocateElem(L,e) //在表L中按照值查找
GetElem(L,i) //按位查找,获取表L中第i个位置的元素的值 
Length(L) //求表的长度
IsEmpty(L) //判断表L是否是空
PrintList(L) //打印表L
    \end{minted}
\end{macbox}

线性表初始化
\begin{macbox}{Lyshmily.Y}
    \begin{minted}{c}
void InitList(SqList &L){
    L.elem = (ElemType *)malloc(LIST_INIT_SIZE * sizeof(ElemType));
    if(!L.elem) exit(OVERFLOW);
    L.length = 0;
    L.listsize = LIST_INIT_SIZE;
}
    \end{minted}
\end{macbox}