\chapterimage{chap1.jpg}
\chapter{绪论}
\section{数据结构}
\begin{definition}[基本术语]
    \begin{enumerate}
        \item 数据: 描述客观事物的符号, 是计算机中可以操作的对象, 是计算机中的输入和输出的信息
        \item 数据元素: 是数据的基本单位, 在计算机中通常作为一个整体进行考虑和处理
        \item 数据项: 一个数据元素可以由若干个数据项组成, 数据项是数据不可分割的最小单位
        \item 数据结构: 是相互之间存在一种或多种特定关系的数据元素的集合
        \item 数据对象: 是具有相同性质的数据元素的集合, 是数据的子集
    \end{enumerate}
\end{definition}

\begin{definition}[数据结构三要素]
    \begin{enumerate}
        \item 逻辑结构: 集合、线性结构(一对一)、树形结构(一对多)、图形结构(多对多)
        \item 物理结构: 顺序存储结构(逻辑和物理都相邻)、链式存储结构(指针表示)、散列存储
        \item 数据运算: 初始化、插入、删除、查找、更改、遍历
    \end{enumerate}
\end{definition}

\begin{definition}[数据类型]
    \begin{enumerate}
        \item 数据类型: 一个值的集合和定义在此集合上的一组操作的总称
    \end{enumerate}
\end{definition}

\section{算法}

\begin{definition}[算法]
    \begin{enumerate}
        \item 算法: 是解决特定问题求解步骤的描述, 是指令的有限序列
    \end{enumerate}
\end{definition}

\subsection{算法的特性}

\subsection{算法的评估}
