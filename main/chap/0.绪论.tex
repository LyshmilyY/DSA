\chapterimage{chap1.jpg}
\chapter{绪论}
\section{数据结构}
\begin{definition}[基本术语]
    \begin{enumerate}
        \item 数据($\mathbf{Data}$):客观事物的符号, 是所有能输入到计算机中并被计算机程序处理的符号的集合.
        \item 数据元素($\mathbf{Data\ Element}$):数据的基本单位, 也被称为元素、记录,数据元素用于完整地描述一个对象, 比如一名学生记录、树中棋盘的一个格局和图中的一个顶点.
        \item 数据项($\mathbf{Data\ Item}$):是组成数据元素的、有独立含义的、不可分割的最小单位. 一个数据元素可以由一个或多个数据项组成. 比如一名学生记录中包含学号、姓名、性别、年龄等数据项.
        \item 数据对象($\mathbf{Data\ Object}$):性质相同的数据元素的集合, 是数据的一个子集. 比如学生集合、图中的顶点集合.
        \item 数据结构($\mathbf{Data\ Structure}$):是相互之间存在一种或者多种特定关系的数据元素的集合
    \end{enumerate}
\end{definition}

\begin{definition}[数据结构三要素]
    \begin{enumerate}
        \item 逻辑结构: 集合、线性结构(一对一)、树形结构(一对多)、图形结构(多对多)
        \item 物理结构: 顺序存储结构(逻辑和物理都相邻)、链式存储结构(指针表示)、散列存储
        \item 数据运算: 初始化、插入、删除、查找、更改、遍历
    \end{enumerate}
\end{definition}

\begin{definition}[数据类型和抽象数据类型]
    \begin{enumerate}
        \item 数据类型($\mathbf{Data\ Type}$): 一个值的集合和定义在此集合上的一组操作的总称
        \item 抽象数据类型($\mathbf{Abstract\ Data\ Type},\mathbf{ADT}$): 是指一个数学模型以及定义在此数学模型上的一组操作, 具体包括数据对象、数据对象上关系的集合以及对数据对象的基本操作的集合
    \end{enumerate}
\end{definition}

\section{算法和算法分析}

\begin{definition}[算法的定义及其特性]
    一、算法: 是解决特定问题求解步骤的描述, 是指令的有限序列

    二、算法的特性:
    \begin{enumerate}
        \item 有穷性: 算法在执行有限步之后终止 
        \item 确定性: 算法的每一步骤都有确切的含义, 不会出现二义性
        \item 可行性: 算法的每一步都是可行的, 可以通过执行有限次完成
        \item 输入: 一个算法有零个或多个输入
        \item 输出: 一个算法有一个或多个输出
    \end{enumerate}
    三、评价算法优劣的标准:
    \begin{enumerate}
        \item 正确性: 算法的输出结果符合要求
        \item 可读性: 算法是容易阅读和理解的
        \item 健壮性: 算法对不合理数据有较好的处理能力
        \item 高效性: 包括时间和空间两个方面, 时间高效指的是算法设计合理, 执行效率高, 空间高效指的是算法执行过程中所需的存储空间最小
    \end{enumerate}
\end{definition}
